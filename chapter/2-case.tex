\chapter{The case in question}

\section{Overview}

We have used the Coq proof assistant to formalize parts of the proofs of the following paper, \citeauthor{mbezem} \cite{mbezem}.
This paper solves two problems that occur in dependent type systems where typings depend
on universe-level constraints. We focused on formalizing the proof of theorem 3.2 from the paper.
Since this proof is complex enough that mistakes are possible, it was a good candidate for formalization.
It also has direct applications to the formalization and verification of the Coq proof assistant itself,
since the algorithm outlined in the proof is being tested \cite{coqgithub}
for use in checking loops in Coqs type system. Hence it should better be correct!

\section{Relevant parts of the paper}

In the paper, join-semilattices with inflationary endomorphisms
are simply called semilattices. An inflationary endomorphism is a function
that maps an element to itself or to a greater element in the ordered set.
A join-semilattice is a partially ordered set in which any two elements have a least upper bound,
called their join.

\textcolor{red}{insert def. of frontier/f here}

A semilattice presentation consists of a set $V$ of generators
(also called variables) and a set  $C$ of constraints
(also called relations).

\textcolor{red}{insert def. of $S_C$ and related notation here}

% A term over V is a term of the form
% $x_1 + k_1 \lor \ldots \lor x_m + k_m$,
% where $x_i \in V$ and $k_i \in N^{\infty}$.

\subsection{Theorem 3.2}

Theorem 3.2 states that for any finite semilattice representation
$(V, C)$ and any function $f : V \rightarrow N^{\infty}$,
the least $g \ge f$ that is a model of $\overline{S_C}$ can be computed.

\subsection{Lemma 3.3}

Theorem 3.2 has a special case that is solved by an additional lemma,
lemma 3.3. This lemma states that given a finite semilattive presentation
$(V, C)$ and a strict subset $W \subset V$, if for any function
$f : W \rightarrow N^{\infty}$, the least $g \ge f$ that is a model of
$\overline{S_C}|W$ can be computed, then for any function
$f : V \rightarrow N^{\infty}$ with $f(V - W) \subseteq N$,
the least $h \ge f$ that is a model of $\overline{S_C}\downarrow W$ can be computed.

Due to time constraints we have given a formalization of this lemma,
but not proven it.
When testing the algorithm generated by our formalization of theorem 3.2,
we have manually edited the code to use the identity function instead of crashing due to
the lack of a proof of lemma 3.3.
