\chapter{The Case in Question}
\label{ch:the_case_in_question}

\section{Overview}
\label{sec:overview}

Recall the three questions posed in the abstract.
Firstly, we want to use the Coq proof assistant to formalize parts of the proofs of the following paper, by \citeauthor{mbezem} \cite{mbezem}.
This paper solves two problems that occur in dependent type systems where typings depend
on universe-level constraints, and the two problems can be summarized as follows:

\begin{enumerate}
    \item The uniform word problem for an an equational theory of semilattices
          with one inflationary endomorphism can be solved in polynomial time.
    \item Loop-checking in a type system with universe-level constraints
          can be done in polynomial time.
\end{enumerate}

From \cite{mbezem}, we get a definition of a uniform word problem for an equational theory $T$:

\begin{quote}
The uniform word problem for an equational theory $T$ is to determine,
given a finite set $E$ of relations between generators,
whether a given relation is provable from $E$ in $T$.
\end{quote}
In our case, the equational theory $T$ is the theory of join-semilattices
with one inflationary endomorphism.
This problem is shown in \cite{mbezem} to be solved in polynomial time,
and provability in our case of a semilattice theory is shown to be equivalent to
provability in Horn clauses.
We will explain in more detail in the next subsection what join-semilattices
inflationary endomorphisms and Horn clauses are.
Loop-checking, in our case, means determining if a set of constraints
on some set of variables generate an infinite loop between constraints.
This is important for algorithms such as a type-checking algorithm where
type universe levels have to be determined, and the algorithm has to be decideable.
We have focused on formalizing the proof of \Cref{thm:3.2} from \cite{mbezem}.
Since this proof is complex enough that mistakes are possible, it is a good candidate for formalization.
By formalizing the proof and then verifying it using Coq, we can be sure that it is correct.

Secondly, we are also interested in finding out whether or not the process of formalizing the proof
can be completed in a reasonable amount of time, and with a reasonable amount of effort.
Is the promise of a complete verification of the proof worth the effort of formalizing it?

Finally, by using an advanced feature of Coq, we can also extract an algorithm from the proof,
where this algorithm is certified to produce correct results.
As an aside, this algorithm has direct applications to the formalization and verification of the Coq proof assistant itself,
as Coq employs a similar algorithm to resolve universe-level constraints in its type system.

\section{Relevant parts of the paper}
\label{sec:relevant_parts_of_the_paper}

We will paraphrase some key concepts from \cite{mbezem} that are needed to understand
our implementation and formalization.

We will call join-semilattices with inflationary endomorphisms simply semilattices.
A \emph{join-semilattice} is a partially ordered set in which any two elements have a least upper bound,
called their join.
The \emph{join} of a semilattice is a binary operation $\lor$ that is commutative, associative and idempotent.
An \emph{inflationary endomorphism}, in our case, is a function
that maps an element to itself or to a greater element in the ordered set.
We denote the inflationary endomorphism as $^+$, for every lattice element $x$,
$x^+$ is also called the successor of $x$.
The fact that $x$ is an endomorphism means that we get the following properties:
$(x \lor y)^+ = x^+ \lor y^+$ and $x \lor x^+ = x^+$.


A \emph{semilattice presentation} consists of a set $V$ of \emph{generators}
(also called \emph{variables}) and a set $C$ of \emph{relations}
(also called \emph{constraints}).
For a semilattice $t$ and $k \in \mathbb{N}$, $t + k$ denotes the k-fold successor of $t$, thus making $t + 0 = t$ and $t + 1 = t^+$.
A term over $V$ has the form $x_1 + k_1 \lor \ldots \lor x_m + k_m$, where $x_i \in V$ and $k_i \in \mathbb{N}$.
Every semilattice expression is equal to a term over $V$.

\emph{Horn clauses} are propositional clauses $A \rightarrow b$, with a non-empty body $A$ and conclusion $b$.
The atoms are of the form $x + k$, where $x \in V, k \in \mathbb{N}$.
We call this special form of Horn clauses simply \emph{clauses}.

A relation is an equation $s = t$, where $s$ and $t$ are terms over $V$.
A constraint (relation), like $a \lor b = c^+$ (with $a, b, c \in V$), expresses a relation between the generators (variables) $a, b, c$.
For each constraint $s = t$, we generate clauses by replacing join by conjunction, denoted by ",", and $\geq$ by implication.
E.g. for the constraint $a \lor b = c^+$, we first derive $a \lor b \geq c^+$ and $c^+ \geq a \lor b$.
From $a \lor b \geq c^+$, we generate the clause $a, b \rightarrow c^+$.
From $c^+ \geq a \lor b$, we derive $c^+ \geq a$ and $c^+ \geq b$.
We then generate the clause $c^+ \rightarrow a$ from the former, and the clause $c^+ \rightarrow b$ from the latter.
This generated set of clauses is denoted by $S_{s=t}$, which in our example makes it
$S_{a \lor b = c^+} := \{a, b \rightarrow c^+, c^+ \rightarrow a, c^+ \rightarrow b\}$.
We define $S_C$ as the union of all $S_{s=t}$, where $s = t$ is a constraint in $C$.
From the axiom $x \lor x^+ = x^+$, we can derive the following clauses:
$x, x^+ \rightarrow x^+$,
$x^+ \rightarrow x^+$ and $x^+ \rightarrow x$.
We are only interested in the last one, which we call a \emph{predecessor clause}.
A predecessor clauses ensures that any atom $x + k$ that is true in a model,
also has $x + k - 1$ true in that model.

% Predecessor clauses are derived from the axiom $x \; \lor \; x^{+} = x^{+}$ and have the form
% $x + k + 1 \rightarrow x + k$, where $(x \in V, k \in \mathbb{N})$

In \cite[Theorem 2.2,~p.~3]{mbezem} it is proven that provability in the semilattice theory
is equivalent to Horn clause provability, making it so that we can continue with the latter in the rest of this thesis.

We define closure under shifting upwards as follows:
if $A \rightarrow b$ is in the set of clauses, then so must $A + 1 \rightarrow b + 1$ be.
$A + 1$ denotes the set of atoms of the form $a + 1$, where $a \in A$.
Given a finite semilattice presentation $(V, C)$,
and a subset $W \subseteq V$, we denote by
$\overline{S_C}$ the smallest set of clauses that is closed under shifting upwards,
by $\overline{S_C} \mid W$ the set of clauses in $\overline{S_C}$
mentioning only variables in $W$, and by $\overline{S_C} \downarrow W$
the set of clauses in $\overline{S_C}$ with conclusion over $W$.

A function $f : V \rightarrow \mathbb{N}^{\infty}$ specifies a downward closed set of atoms,
namely $\{v + k \mid v \in V, k \in \mathbb{N}, k \le f(v)\}$,
where $\mathbb{N}^{\infty}$ is the set of natural numbers extended with $\infty$.
This set contains all atoms $v + k$ if $f(v) = \infty$.
The sets specified by such functions are models of $\overline{S_C}$.

\subsection{Lemmas and theorems}
\label{ssec:lemmas_and_theorems}

\begin{customlem}{Lemma 3.1}[{\cite[p.~3]{mbezem}}]
\label{lem:3.1}
Given $f : V \rightarrow \mathbb{N}^{\infty}$,
and a clause $A \rightarrow b$, let $P$ be the problem whether or not
$A + k \rightarrow b + k$ is satisfied by $f$ for all $k \in \mathbb{N}$.
Then $P$ is decidable.
\end{customlem}

The proof of \Cref{lem:3.1} demonstrates that the problem $P$
is decidable, meaning we can indeed write an algorithm that
determines whether or not the problem holds for all $k \in N$.
\Cref{lem:3.1} is also crucial for making case distinctions
in further proofs, since we know that any $S_C$ is finite.

\begin{customthm}{Theorem 3.2}[{\cite[p.~3]{mbezem}}]
\label{thm:3.2}
Let $(V, C)$ be a finite semilattice presentation.
For any $f : V \rightarrow \mathbb{N}^{\infty}$
we can compute the least $g \ge f$ that is a model of $\overline{S_C}$.
\end{customthm}

\Cref{thm:3.2} has a special case that is solved by an additional lemma:

\begin{customlem}{Lemma 3.3}[{\cite[p.~3]{mbezem}}]
\label{lem:3.3}
Let $(V, C)$ be a finite semilattice presentation.
Let $W$ be a strict subset of $V$ such that for any
$f : W \rightarrow \mathbb{N}^{\infty}$ we can
compute the least $g \geq f$ that is a model of $\overline{S_C} \mid W$.
Then for any $f : V \rightarrow \mathbb{N}^{\infty}$ with $f(V - W) \subseteq \mathbb{N}$,
we can compute the least $h \geq f$ that is a model of $\overline{S_C} \downarrow W$.
\end{customlem}
