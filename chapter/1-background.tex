\chapter{Background}

\section{Formalizing Mathematical Problems}

\subsection{Proofs}

When solving mathematical problems, one often uses proofs
to justify some claim. We can group proofs into two types; \textit{informal} and \textit{formal} proofs.

\subsubsection{Informal proofs}
An informal proof is often written in a natural language, and the proof is adequate if most
readers are convinced by the proof \cite{bpierce}.
As a proof grows larger and more complex, it becomes harder to follow, which can
ultimately lead to errors in the proof's reasoning. This might cause the whole proof
to be incorrect \cite{rkhamsi}.

\subsubsection{Formal proofs}
A formal proof is written in a formal language, and can be compared to a computer program
written in a programming language. Writing a formal proof is more difficult than writing an informal proof.

\section{Type theory}

Type theory groups mathematical objects with similar properties together by assigning them a "type".
Similarily to data types in computer programming, we can use types to represent mathematical
objects. For example, we can use the data type \texttt{nat} to represent natural numbers.

\subsection{Propositions as types}
The concept of propositions as types sees the proving of a mathematical proposition
as the same process as constructing a value of that type.
For example, to prove a proposition $P$ which states "all integers are the sum of four squares",
we must construct a value of the type $P$ that shows that this is true for all integers.
Proofs are mathematical objects; thus a proposition can be viewed as having the type of all its proofs (if any!).
We can use this correspondance to model a proof as a typed computer program.
The power of this concept comes from the fact that we can use a type checker to verify that
our program is typed correctly, and thus that the corresponding proof is valid.


\section{Proof assistants}

Using a \textit{proof assistant}, we can verify a formal proof mechanically.

\subsection{Coq}
\textit{Coq} is an example of a proof assistant.
Coq uses type theory to formulate and verify proofs, but can also be used as a functional programming language \cite{cintro}.

\subsection{Extraction of programs from verified proofs}

Coq also enables us to extract and execute programs from our proofs, once they have been verified.

\subsection{Agda}

\subsection{Isabelle}

\subsection{Lean}

\subsection{Higher-Order Logic}
