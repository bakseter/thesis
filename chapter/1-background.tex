\chapter{Background}

% https://www.ams.org/notices/200811/200811FullIssue.pdf

\section{Formalizing Mathematical Problems}

\subsection{Proofs}

When solving mathematical problems, we often use proofs
to either \textbf{warrant} a claim or to \textbf{explain} why the claim is true.
We can group proofs into two types; \textit{informal} and \textit{formal} proofs.

Traditional informal proofs emphasize the latter, and they take for grated that this is sufficent from the former.
Formal proofs on the other hand, ??? on the former; depending on the formalism used they have some
explanatory merit as well.

An informal proof is often written in a natural language, and the proof is adequate if most
readers are convinced by the proof \cite{bpierce}.

As proofs grows larger and more complex, they becomes harder to follow, which can
ultimately lead to errors in the proofs' reasoning. This might cause the whole proof
to be incorrect \cite{rkhamsi}, and even the result of the proof might be wrong.

A formal proof is written in a formal language, and can be compared to a computer program
written in a programming language. Writing a formal proof is more difficult than writing an informal proof.

\section{Type theory}

Type theory groups mathematical objects with similar properties together by assigning them a "type".
Similarily to data types in computer programming, we can use types to represent mathematical
objects. For example, we can use the data type \texttt{nat} to represent natural numbers.

\subsection{Propositions as types}

% https://homepages.inf.ed.ac.uk/wadler/papers/propositions-as-types/propositions-as-types.pdf

The concept of propositions as types sees proving a mathematical proposition
as the same process as constructing a value of that type.
For example, to prove a proposition $P$ which states "all integers are the sum of four squares",
we must construct a value of the type $P$ that shows that this is true for all integers.
Such a value is a function that for any input $n$ returns a proof that $n$ is the sum of four squares.
Proofs are mathematical objects; thus a proposition can be viewed as having the type of all its proofs (if any!).
We can use this correspondance to model a proof as a typed computer program.
The power of this concept comes from the fact that we can use a type checker to verify that
our program is typed correctly, and thus that the corresponding proof is valid.


\section{Proof assistants}

Using a \textit{proof assistant}, we can get computer support for continuity and verify a formal proof mechanically.

\subsection{Coq}
\textit{Coq} is an example of a proof assistant.
Coq uses type theory to formulate and verify proofs, but can also be used as a functional programming language \cite{cintro}.

\subsection{Agda}

\subsection{Isabelle}

\subsection{Lean}

\subsection{Higher-Order Logic}

\section{Extraction of programs from verified proofs}
