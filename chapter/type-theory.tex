\chapter{Type theory}

Type theory groups mathematical objects with similar properties together by assigning them a "type".
Similarily to data types in computer programming, we can use types to represent mathematical
objects. For example, we can use the data type \texttt{nat} to represent natural numbers.

\section{Propositions as types}
The concept of propositions as types sees the proving of a mathematical proposition
as the same process as constructing a value of that type.
For example, to prove a proposition $P$ which states "all integers are the sum of four squares",
we must construct a value of the type $P$ that shows that this is true for all integers.
Proofs are mathematical objects; thus a proposition can be viewed as having the type of all its proofs (if any!).
We can use this correspondance to model a proof as a typed computer program.
The power of this concept comes from the fact that we can use a type checker to verify that
our program is typed correctly, and thus that the corresponding proof is valid.

\section{Extraction of programs from verified proofs}

Coq also enables us to extract and execute programs from our proofs, once they have been verified.
