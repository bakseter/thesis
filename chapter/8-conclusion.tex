\chapter{Conclusion}

In this thesis, we have presented a detailed exploration into formalizing
the proofs from two problems in dependent type theory,
and extracting a certified program from their verified formal proofs using Coq.
We have investigated the utility of proof assistants and dependent type theory
in providing a structured approach to proofs and formalization,
which ultimately led to a (semi-)practical application
in the form of a certified correct algorithm.

We started by introducing relevant theoretical concepts,
including propositions as types and dependent types.
We then used these to explore the Coq proof assistant,
and compared it to other available tools.

The case presented in \Cref{ch:the_case_in_question}
provided an opportunity to utilize these theoretical concepts in a practical setting.
Simplifications were made to fit the time constraints of the project.
Since reasoning about sets were a sizeable part of the theorem we formalized,
a good implementation of sets was important.
Various implementations were considered, we made a choice that best fit our needs.
We then presented the building blocks of our formalization,
in the form of various data types, functions, predicates, lemmas and theorems.

The practical relevance of the formalization effort was demonstrated through the extraction
of our Coq proofs into a certified Haskell program.
We gave an example of how the program could be run to solve a simple case,
but also discussed the limitations of the algorithm.
We also gave a more sizable "real-world" example, which again
demonstrated the practicality of our work.

We then assessed how trough our formalization effort,
we gained more confidence in the correctness of the original informal proofs posed in \cite{mbezem}.
Furthermore, the extracted algorithm was evaluated for its usefulness,
highlighting the practical benefits of our formalization.
We then explained how our work could be extended in the future,
namely by completing the proofs of remaining logical lemmas,
providing a formal proof of \Cref{lem:3.3}, and proving the minimality of the model generated by \Cref{thm:3.2}.

In conclusion, we have in this thesis successfully presented
a well-founded journey from theoretical understanding to practical application.
The power of proof assistants and type theory was showcased,
opening the doors for further exploration and development in this
fascinating intersection of logic, mathematics, and computer science.
