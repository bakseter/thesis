\chapter{Conclusion}

In this thesis, we started by introducing relevant theoretical concepts,
including propositions as types and dependent types.
We then used these concepts to explore the Coq proof assistant,
and compared it to other related tools.

The case presented in \Cref{ch:the_case_in_question}
provided an opportunity in using Coq to apply these theoretical concepts
in a more practical setting.
Simplifications were made to fit the time constraints of the project.
Since reasoning about sets were a sizable part of the theorem we formalized,
a good implementation of sets was important.
Various implementations were considered, and we made a choice that best fit our needs.
We then presented the "building blocks" of our formalization,
in the form of various data types, functions, predicates, lemmas and theorems.

The relevance of our project was demonstrated through the extraction
of our Coq proofs into a certified Haskell program.
We gave an example of how the program could be run to solve a simple case,
but also discussed the limitations of the algorithm.
We then gave a more sizable "real-world" example,
which would determine if our extracted algorithm could function as a prototype.

We then assessed how through our formalization effort,
we gained more confidence in the correctness of the original informal proofs posed in \cite{mbezem}.
Furthermore, the extracted algorithm was evaluated for its usefulness as a protype,
and we discussed what useful value is provided with a formal proof and resulting extraction.

We then explained how our work could be extended in the future,
namely by completing the proofs of remaining logical lemmas,
providing a formal proof of \Cref{lem:3.3}, and proving the minimality of the model generated by \Cref{thm:3.2}.

In conclusion, we have in this thesis presented a detailed exploration into formalizing
the proofs from two problems in dependent type theory,
and extracting a certified program from their verified formal proofs using Coq.
We have investigated the utility of proof assistants and dependent type theory
in providing a structured approach to proofs and formalization,
which ultimately led to a practical application,
in the form of a certified correct program useable as a prototype.
