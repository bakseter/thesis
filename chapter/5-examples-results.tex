\chapter{Examples \& Results}
\label{chap:examples_results}

\section{Examples using the extracted Haskell code}
\label{sec:examples}

\subsection{Defining examples for extraction in Coq}
\label{sec:defining_examples}

It is easiest to define as much of the example as possible in Coq and then extract it to Haskell,
since Coq heavily prioritizes code correctness over readability when extracting.
This makes much of the Haskell code hard to read due to unconventional syntax,
e.g. using a recursion operator instead of calling a function recursively.
First, we can look at the type of \lstinline{thm_32} in Coq to see what arguments we need to provide it with.
We do this by using the Coq command \lstinline{Check}.

\begin{minipage}{\linewidth}
\begin{lstlisting}[language=Coq, label={lst:thm_32_type_sig_coq}, caption={Type signature of \lstinline{thm_32} in Coq}]
thm_32
     : forall (n m : nat) (Cs : set Clause) (V W : set string) (f : Frontier),
       pre_thm n m Cs V W f
\end{lstlisting}
\end{minipage}

We see that the type of \lstinline{thm_32} is a proof of \lstinline{pre_thm}, with the quantified variables
\lstinline{n, m, Cs, V, W, f}.
We can now look at the type signature of \lstinline{pre_thm}.

\begin{minipage}{\linewidth}
\begin{lstlisting}[language=Coq, label={lst:pre_thm_type_sig_coq}, caption={Type signature of \lstinline{pre_thm} in Coq}]
pre_thm
     : nat ->
       nat -> set Clause -> set string -> set string -> Frontier -> Set
\end{lstlisting}
\end{minipage}

We see that \lstinline{pre_thm} returns an object in \lstinline{Set}.
This object is the proposition defined in \Cref{lst:pre_thm_def},
where all the variables are the arguments given to \lstinline{pre_thm}, i.e. the variables quantified in \lstinline{thm_32}.
This proposition states some logical assumptions, and concludes with a definition of \lstinline{ex_lfp_geq}.
By the notion of propositions as types, this proposition is a function
that transforms a proof of the assumptions into a proof of the conclusion.
Since the conclusion is a definition of \lstinline{ex_lfp_geq} instantiated with some variables,
we get a proof of \lstinline{ex_lfp_geq} for these same variables.
Looking at the definition of \lstinline{ex_lfp_geq},
it defines a proposition stating that there exists a \lstinline{g : Frontier} such that the proposition holds.
If such a \lstinline{g} exists, then the \lstinline{g} itself is evidence (proof) that the proposition holds.
Hence, a proof of \lstinline{ex_lfp_geq} is simply a \lstinline{Frontier} that satisfies the conditions in \lstinline{ex_lfp_geq}.

With all this information, we can now define an example of \lstinline{thm_32} in Coq.

\begin{minipage}{\linewidth}
\begin{lstlisting}[language=Coq, label={lst:thm_32_coq_example}, caption={\lstinline{thm_32} example}]
(* represents the clauses: a -> b + 1, b + 1 -> c + 2 *)
Example Cs := [
  ["a" & 0] ~> "b" & 1;
  ["b" & 1] ~> "c" & 2
].

(* all-zero frontier *)
Example thesis_ex_1_f := frontier_fin_0.

(* extract only variables from clauses *)
Example vars' := nodup string_dec (vars Cs).

(* (partially) apply arguments to thm_32 *)
Example thesis_ex_1 :=
  thm_32
  (Datatypes.length vars')
  (Datatypes.length vars')
  Cs
  vars'
  []
  thesis_ex_1_f.
\end{lstlisting}
\end{minipage}

Since Coq eliminates many of the logical parts of the proof when extracting,
\cite[p. ~8]{coqextroverview},
we can avoid the tedious task of proving all the assumptions of \lstinline{pre_thm} by simply
using the extracted Haskell function for \lstinline{thesis_ex_1}.
The final assumption of \lstinline{pre_thm}, namely \lstinline{ex_lfp_geq Cs (nodup string_dec W) (nodup string_dec W) f},
will however not be removed by Coq, since it is an object in \lstinline{Set}.
This assumption is trivially true for any \lstinline{f}, since \lstinline{W} is an empty list in our example.
Since Coq simplifies the type of \lstinline{ex_lfp_geq} to \lstinline{Frontier},
we can just include the same frontier \lstinline{f} as in \lstinline{thesis_ex_1} to the extracted Haskell
version of \lstinline{thesis_ex_1}, as a replacement for a proof of the above trivially true assumption.
We then receive a \lstinline{Frontier} as output, which is the result of the theorem;
the frontier that is a model of a set of clauses \lstinline{Cs}.
This frontier can then be applied to any string representing a variable to get the value of that variable in the model,
which should be either a natural number or infinity.

\subsection{Necessary alterations to the extracted Haskell code}
\label{sec:necessary_alterations}

Since we have not given a proof of \Cref{lem:3.3}, Coq will include a "placeholder" definition of the lemma in the extracted code.
This definition will simply call the \lstinline{error} function, which will crash the program when called.
We circumvent this by replacing the extracted definition of \lstinline{lem_33} with the identity function
for any frontier. We will look at some examples where this workaround is not sufficient in
\Cref{sec:limitations}.

If we want to actually read the output from the extracted functions, we also need to
derive a \lstinline{Show} instance for \lstinline{Ninfty}.
What this means is that we need to define a function \lstinline{show :: Ninfty -> String},
which will be used by Haskell to convert a \lstinline{Ninfty} to a \lstinline{String}.
This can be done by simply adding the line \lstinline{deriving Prelude.Show} to the definition of \lstinline{Ninfty},
which will make Haskell just print the constructors of \lstinline{Ninfty},
which will be either \lstinline{Fin n} for some natural number \lstinline{n},
or \lstinline{Infty} for infinity.

\subsection{Example output}
\label{sssec:example_output}

We can now run the example from \Cref{lst:thm_32_coq_example} using GHCi,
which is an interactive Haskell interpreter that is included with GHC, the standard Haskell compiler.

\begin{minipage}{\linewidth}
\begin{lstlisting}[language=bash, label={lst:thm_32_haskell_output}, caption={\lstinline{thm_32} example output}]
ghci> (thesis_ex_1 thesis_ex_1_f) "a"
Fin 0
ghci> (thesis_ex_1 thesis_ex_1_f) "b"
Fin 1
ghci> (thesis_ex_1 thesis_ex_1_f) "c"
Fin 2
ghci> (thesis_ex_1 thesis_ex_1_f) "x"
Fin 0
\end{lstlisting}
\end{minipage}

The expression \lstinline[language=Haskell]{(thesis_ex_1 thesis_ex_1_f)} produces a function that is the minimal model of \lstinline{Cs}.
When given a string value (variable) from the set of clauses, the function will compute the value of that variable.
When given any other variable, the function will return the value that the original frontier given as input
would return for that variable, which is always \lstinline{Fin 0} in this case
(since in our example the frontier is \lstinline{frontier_fin_0},
which is a constant function that always returns \lstinline{Fin 0}).

\section{Real world example}
\label{sec:real_world_example}

As stated previously, the algorithm described \Cref{thm:3.2} is being
tested for use in checking loops and determining type universe levels in the
type system of Coq.
We briefly explained type universes in the case of Coq in \Cref{ssec:ex_lfp_geq}.
The error that occurred in the example of a universe inconsistency in \Cref{lst:universe_inconsistency}
was detected by Coq's own universe-constraint checking algorithm;
we can also use our algorithm to detect these universe inconsistencies.

Using the Coq command \lstinline{Print Universes} we can see the current state of the universe hierarchy for any Coq file.
Doing this at the bottom of our file \lstinline{Main.v}, after our proof of \Cref{thm:3.2},
we get the following output (truncated):

\begin{minipage}{\linewidth}
\begin{lstlisting}[language=Coq, label={lst:universe_hierarchy}, caption={Universe hierarchy in \lstinline{Main.v}}]
DefaultRelation.u0 <= Coq.Relations.Relation_Definitions.1 ;
default_relation.u0 <= DefaultRelation.u0 ;

...

Coq.Structures.Equalities.1 <= Morphisms.proper_sym_impl_iff_2.u1 ;
\end{lstlisting}
\end{minipage}

This output has over 5000 lines, so we have truncated it to only show the first and last few lines.
We can translate these constraints into a set of clauses \lstinline{Cs} as such:

\begin{minipage}{\linewidth}
\begin{lstlisting}[language=Coq, label={lst:universe_hierarchy_clauses}, caption={Universe hierarchy as clauses}]
Definition Cs := [
    ["Coq.Relations.Relation_Definitions.1" & 0]
        ~> "DefaultRelation.u0" & 0;
    ["DefaultRelation.u0" & 0]
        ~> "default_relation.u0" & 0;

    ...

    ["Morphisms.proper_sym_impl_iff_2.u1" & 0]
        ~> "Coq.Structures.Equalities.1" & 0
].
\end{lstlisting}
\end{minipage}

We can then run our algorithm on these clauses to get a minimal model of \lstinline{Cs}.
This minimal model is a function that assigns to each type universe a natural number,
which corresponds to its universe level. It is minimal in the sense that no other mapping from
type universes to universe levels that satisfies the constraints in \lstinline{Cs} has a lower
universe level for any given type universe.
Even though we have not given a formal proof of \Cref{lem:3.3},
this "real world" example still gives a correct minimal model for the type universe constraints from Coq.

\subsection{Loose comparison to Coq's universe consistency checker}

To give the reader an idea of how our extracted algorithm compares to some of
the work done by the Coq compiler on a similar problem, we can
loosely compare the run time of our algorithm to the run time of
a full Coq compilation of our Coq project, which includes all
lemmas and theorems needed for \Cref{thm:3.2}, and the proof itself.
It is important to note that this is \textbf{not} a one-to-one comparison,
since Coq does a lot more when compiling than simply checking universe constraints.
We give this comparison only to give the reader an idea of the order of magnitude of the run time of our algorithm
in a real world setting.

We compile our Haskell algorithm using the Stack build tool,
which is more efficient than using an interpreter such as GHCi.
Using the \lstinline[language=bash]{time} command, we can check how long
any command takes. For our Haskell algorithm we run the command \lstinline[language=bash]{stack run},
and for the Coq example we run the command \lstinline[language=bash]{make -f CoqMakefile}.

\begin{minipage}{\linewidth}
\begin{lstlisting}[language=bash, label={lst:real_world_example_bench}, caption={Real-world example benchmark}]
stack run  18.47s user 0.94s system 100% cpu 19.369 total
make -f CoqMakefile  3.25s user 0.76s system 99% cpu 4.045 total
\end{lstlisting}
\end{minipage}

As we can see, our algorithm takes around 18 seconds, while a full Coq compilation
takes around 3 seconds.
Again, keep in mind that Coq does a lot more than just checking universe constraints when
compiling. But, as we can see from these results, our extracted algorithm
is not too slow to be used as a prototype.

\section{Extension of \Cref{lst:thm_32_coq_example} that fails}
\label{sec:limitations}

There are some cases where the extracted Haskell code will crash.
To show this, we can extend the set of clauses from \Cref{lst:thm_32_coq_example} as such:

\begin{minipage}{\linewidth}
\begin{lstlisting}[language=Coq, label={lst:thm_32_coq_example_extended}, caption={\lstinline{thm_32} example extended}]
Example Cs := [
  ["a" & 0] ~> "b" & 1;
  ["b" & 1] ~> "c" & 2;
  (* add new clause *)
  ["c" & 2] ~> "d" & 3
].

Example thesis_ex_2_f := frontier_fin_0.
Example vars' := nodup string_dec (vars Cs).

Example thesis_ex_2 :=
  thm_32
  (Datatypes.length vars')
  (Datatypes.length vars')
  Cs
  vars'
  []
  thesis_ex_2_f.
\end{lstlisting}
\end{minipage}

If we try to run the extracted Haskell code for this example,
we will always encounter a runtime exception:

\begin{minipage}{\linewidth}
\begin{lstlisting}[language=bash, label={lst:thm_32_haskell_output_extended}, caption={Output of \lstinline{thm_32} example extended}]
ghci> (thesis_ex_2 thesis_ex_2_f) "a"
*** Exception: absurd case
CallStack (from HasCallStack):
  error, called at ex.hs:13:3 in main:Ex
\end{lstlisting}
\end{minipage}

The exception "absurd case" is what happens when a logical assumption in the proof
is not correct, which is the case here.
To see why this happens, we can do some minor alterations to the extracted Haskell code,
and print out the values of some variables during the execution of the algorithm.
What we want to show is every primary and secondary induction call (\lstinline{IHn} and \lstinline{IHm} respectively),
and the values of the variables \lstinline{n}, \lstinline{m}, \lstinline{V} and \lstinline{W} at the time of those calls.
Doing so, we get the following output (truncated for brevity):

\begin{minipage}{\linewidth}
\begin{lstlisting}[language=bash, label={lst:thm_32_haskell_output_extended_debug}, caption={Output of \lstinline{thm_32} example extended, with debug}]
ghci> (thesis_ex_2 thesis_ex_2_f) "a"
IHn: n = 3, m = 4, V = ["a","b","c","d"], W = []

...

IHn: n = 0, m = 1, V = ["a","b","c","d"], W = []
IHm: n = 1, m = 0, V = [], W = []
IHm: n = 3, m = 0, V = ["b","c","d"], W = ["d"]
*** Exception: absurd case
CallStack (from HasCallStack):
  error, called at ex.hs:13:3 in main:Ex
\end{lstlisting}
\end{minipage}

The debug line we are most interested in is the last one,
which is the last secondary induction call before the exception.
If we refer to lines 133-137 of \Cref{lst:thm_32_full_proof},
we see that the proof asserts (and proves) the following claim:

\begin{minipage}{\linewidth}
\begin{lstlisting}[language=Coq, label={lst:thm_32_false_claim}, caption={False claim in \lstinline{thm_32} proof}]
Datatypes.length
  (set_diff string_dec V (set_union string_dec (nodup string_dec W) U)) <
Datatypes.length
  (set_diff string_dec V (nodup string_dec W)) <= S m.
\end{lstlisting}
\end{minipage}

This claim is false, as we can see by looking at the values of
\lstinline{V}, \lstinline{W} and \lstinline{m} at the time of the last induction call.
We replace the variables in the claim with their values, and reduce (parts of) the claim to the following:

\tabto{2em}
\lstinline{Datatypes.length (set_diff string_dec V (nodup string_dec W)) <= S m}

$\Rightarrow$
\tabto{2em}
\lstinline{Datatypes.length (set_diff string_dec ["b","c","d"] ["d"]) <= S 0}.

$\Rightarrow$
\tabto{2em}
\lstinline{Datatypes.length ["c","b"] <= S 0}.

$\Rightarrow$
\tabto{2em}
\lstinline{2 <= 1}.

The final claim is obviously not true, hence why the Haskell code crashes with "absurd case".
But why does this happen at all? Shouldn't the Coq proof have failed at the point where the claim was asserted?
This error is due to the fact that we have not given a formal proof of \Cref{lem:3.3}.
Since we do not use \Cref{lem:3.3} to calculate a new frontier in the algorithm but instead just
redefine \Cref{lem:3.3} as the identity function, this has consequences for later steps in the algorithm.
This example in particular is "constructed to fail"; as we saw in the previous section,
the algorithm does work for many cases even without a formal proof of \Cref{lem:3.3}.
