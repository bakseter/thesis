\chapter{Implementation}

\section{Choice of set implementation}

The simplest set (or set-like) implementation in Coq are the \textbf{List} and \textbf{ListSet} modules.
These require minimal knowledge of advanced Coq syntax and behave like lists, making proofs by induction easy.
They are also polymorphic, meaning ease of use when making sets of different or self-defined types.
Because of these reasons, we chose to go with \textbf{List} and \textbf{ListSet}.

\section{The Basics}
\subsection{Atom, Clause and Frontier}

The paper \cite{mbezem} uses heavily Horn clauses, which it (and we) simply call clauses.
Following the definition of a Horn clause, a clause contains a body of a set of atomic formulas,
or atoms, and a single atom as the head \cite{halbert}.

We also define the atoms in the clauses as containing one string and one natural number,
since this is sufficient for our implementation.

\begin{minipage}{\linewidth}
\begin{lstlisting}[language=Coq, label={lst:atom_clause_def}, caption={Def. of \lstinline{Atom} and \lstinline{Clause} in Coq}]
Inductive Atom : Type :=
  | atom : string -> nat -> Atom.

Notation "x & k" := (atom x k) (at level 80).

Inductive Clause : Type :=
  | clause : set Atom -> Atom -> Clause.

Notation "ps ~> c" := (clause ps c) (at level 81).
\end{lstlisting}
\end{minipage}

Note also the \lstinline{Notation}-syntax, which allows us to define a custom notation,
making the code easier to read. The expression on the left-hand side of the \lstinline{:=} in quotation marks
is equivalent to the expression on the right-hand side in parentheses.
The level determines which notation should take precedence, with a higher level equaling a higher precedence.

We also want to model functions of the form $f : V \rightarrow \mathbb{N}^{\infty}$, where $V$ is the set of strings (variables)
and $\mathbb{N}^{\infty}$ is the set of natural numbers $\mathbb{N}$
extended by $\infty$, totally ordered by $n < \infty$ for all $n \in \mathbb{N}$.

We implement this in Coq using two types, \lstinline{Ninfty} and \lstinline{Frontier}.
\lstinline{Ninfty} is either a natural number or infinity.
\lstinline{Frontier} is a function from a string (variable) to \lstinline{Ninfty}.

\begin{minipage}{\linewidth}
\begin{lstlisting}[language=Coq, label={lst:ninfty_frontier_def}, caption={Def. of \lstinline{Ninfty} and \lstinline{Frontier} in Coq}]
Inductive Ninfty : Type :=
  | infty : Ninfty
  | fin   : nat -> Ninfty.

Definition Frontier := string -> Ninfty.
\end{lstlisting}
\end{minipage}

Using these definitions of \lstinline{Atom}, \lstinline{Clause} and \lstinline{Frontier},
we can define functions that check whether any given atom or clause is satisfied for any frontier.

\begin{minipage}{\linewidth}
\begin{lstlisting}[language=Coq, label={lst:atom_clause_true_def}, caption={Def. of \lstinline{atom_true} and \lstinline{clause_true} in Coq}]
Definition atom_true (a : Atom) (f : Frontier) : bool :=
  match a with
  | (x & k) =>
    match f x with
    | infty => true
    | fin n => k <=? n
    end
  end.

Definition clause_true (c : Clause) (f : Frontier) : bool :=
  match c with
  | (conds ~> conc) =>
    if fold_right andb true (map (fun a => atom_true a f) conds)
    then (atom_true conc f)
    else true
  end.
\end{lstlisting}
\end{minipage}

The infix function \lstinline{<=?} is the boolean (and decidable) version of the
Coq function \lstinline{<=}, which uses \lstinline{Prop} and is not decidable without
additional lemmas/work/proofs \textcolor{red}{(wording?)}.

We can also define functions that "shift" the number value of atoms or whole clauses by some amount \lstinline{n : nat}.

\begin{minipage}{\linewidth}
\begin{lstlisting}[language=Coq, label={lst:shift_atom_clause_true}, caption={Def. of \lstinline{shift_atom} and \lstinline{shift_clause} in Coq}]
Definition shift_atom (n : nat) (a : Atom)  : Atom :=
  match a with
  | (x & k) => (x & (n + k))
end.

Definition shift_clause (n : nat) (c : Clause) : Clause :=
  match c with
  | conds ~> conc =>
    (map (shift_atom n) conds) ~> (shift_atom n conc)
  end.
\end{lstlisting}
\end{minipage}

Using these definitions, we can now define an important property that will be used
later; whether a set of clauses is true for any shift of \lstinline{n : nat}.

\begin{minipage}{\linewidth}
\begin{lstlisting}[language=Coq, label={lst:all_shifts_true}, caption={Def. of \lstinline{all_shifts_true}}]
Definition all_shifts_true (c : Clause) (f : Frontier) : bool :=
  match c with
  | (conds ~> conc) =>
      match conc with
      | (x & k) =>
          match f x with
          | infty => true
          | fin n => clause_true (shift_clause (n + 1 - k) c) f
          end
      end
  end.
\end{lstlisting}
\end{minipage}

\section{Model}

\subsection{\lstinline{sub_model}}

Given any set of clauses and a function assigning values to the variables,
we can determine if this is a valid model \textcolor{red}{(reword?)}.

We translate this propery to Coq as the recursive function \lstinline{sub_model}.
We have two additional arguments \lstinline{V} and \lstinline{W}; these are
the set of variables (strings) from the set of clauses, and all changed variables (expand on this), respectively.
The function \lstinline{vars_set_atom} simply returns all the variables used
in a set of atoms as a set of strings.

\begin{minipage}{\linewidth}
\begin{lstlisting}[language=Coq, label={lst:sub_model_def}, caption={Def. of \lstinline{sub_model}}]
Fixpoint sub_model (Cs : set Clause) (V W : set string) (f : Frontier) : bool :=
  match Cs with
  | []      => true
  | (l ~> (x & k)) :: t  =>
    (negb (set_mem string_dec x W) ||
     negb (
       fold_right andb true
         (map (fun x => set_mem string_dec x V) (vars_set_atom l))
     ) ||
     all_shifts_true (l ~> (x & k)) f
    ) && sub_model t V W f
  end.
\end{lstlisting}
\end{minipage}

\subsection{\lstinline{geq}}

We want to determine whether the all the values assigned to a set of variables
from one frontier are greater than or equal to all the values assigned to a set of variables
from another frontier. The values are of the type \lstinline{Ninfty}, and the function
only returns true if \textbf{all} the values from the first frontier are greater than
the values from the second frontier.

\begin{minipage}{\linewidth}
\begin{lstlisting}[language=Coq, label={lst:geq_def}, caption={Def. of \lstinline{geq}}]
Fixpoint geq (V : set string) (g f : Frontier) : bool :=
  match V with
  | []      => true
  | h :: t  =>
    match g h with
    | infty => geq t g f
    | fin n =>
        match f h with
        | infty => false
        | fin k => (k <=? n) && geq t g f
        end
    end
  end.
\end{lstlisting}
\end{minipage}

\subsection{\lstinline{ex_lfp_geq}}

We can now combine \lstinline{sub_model} and \lstinline{geq} to construct a lemma
stating that there exists a frontier \lstinline{g} that is a model of the set of clauses \lstinline{Cs}
and is greater than or equal to another frontier \lstinline{f}.

\begin{minipage}{\linewidth}
\begin{lstlisting}[language=Coq, label={lst:ex_lfp_geq_def}, caption={Multiple defs. of \lstinline{ex_lfp_geq}}]
Lemma ex_lfp_geq_P (Cs : set Clause) (V W : set string) (f : Frontier) : Prop :=
  exists g : Frontier, geq V g f = true /\ sub_model Cs V W g = true.

Definition ex_lfp_geq_T (Cs : set Clause) (V W : set string) (f : Frontier) : Type :=
  sig (fun g : Frontier => prod (geq V g f = true) (sub_model Cs V W g = true)).

(* we can also use Set, this def. is equivalent to the def. above *)
Definition ex_lfp_geq_S (Cs : set Clause) (V W : set string) (f : Frontier) : Set :=
  sig (fun g : Frontier => prod (geq V g f = true) (sub_model Cs V W g = true)).
\end{lstlisting}
\end{minipage}

One thing to note here is that we can define this lemma either as a \lstinline{Prop} or as a \lstinline{Type}.
When using \lstinline{Prop}, we define it as a proposition, using standard FOL syntax.

When using \lstinline{Type}, we define it as a type,
using the \lstinline{sig} type constructor in place of \lstinline{exists}.
We also use the \lstinline{prod} type constructor to represent the conjunction of two propositions.

Another thing to note is the difference between \lstinline{Lemma} and \lstinline{Definition}.
The former is used to define a proposition, while the latter is (usually) used to define a type or a non-recursive function.
In this case, we could actually use \lstinline{Definition} instead of \lstinline{Lemma},
but not the other way around.

The reason for defining \lstinline{ex_lfp_geq} as a \lstinline{Type}, is that we can then use
Coq's extraction feature to generate Haskell code from the Coq definitions.
Since \lstinline{ex_lfp_geq} plays a central part in the proof of the main theorem,
it must be defined as a \lstinline{Type} (or as a \lstinline{Set}!) to avoid universe inconsistencies.

\section{Theorem 3.2}

We can now formulate the main theorem, which is theorem 3.2 from the paper \cite{mbezem}.

\begin{minipage}{\linewidth}
\begin{lstlisting}[language=Coq, label={lst:thm_32}, caption={Theorem 3.2 in Coq}]
Definition thm_32 :
  forall Cs : set Clause,
  forall n m : nat,
  forall f : Frontier,
  forall V W : set string,
    incl W V ->
    Datatypes.length (nodup string_dec V) <= n ->
    Datatypes.length (
      set_diff string_dec (nodup string_dec V) (nodup string_dec W)
    ) <= m <= n ->
    ex_lfp_geq Cs (nodup string_dec W) (nodup string_dec W) f ->
    ex_lfp_geq Cs (nodup string_dec V) (nodup string_dec V) f.
Proof.
    (* ... *)
Defined.
\end{lstlisting}
\end{minipage}

The type annotations for the universal quantifiers are added for clarity; they could all be combined
into to one \lstinline{forall} and the type annotiations could be removed.
Note the use of \lstinline{Definition} instead of \lstinline{Lemma} or \lstinline{Theorem}.
This is for the same reason as with \lstinline{ex_lfp_geq}, namely that we want to be able to extract
Haskell code from the definition.
