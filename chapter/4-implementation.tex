\chapter{Implementation of Logical Notions}

\section{Data types}

As discussed in \autoref{chap:case}, we want to represent clauses as
a set of atoms as premises and a single atom as a conclusion.
We implement this in Coq using two types, \lstinline{Atom} and \lstinline{Clause}.

\begin{minipage}{\linewidth}
\begin{lstlisting}[language=Coq, label={lst:atom_clause_def}, caption={\lstinline{Atom} and \lstinline{Clause} in Coq}]
Inductive Atom : Type :=
  | atom : string -> nat -> Atom.

Notation "x & k" := (atom x k) (at level 80).

Inductive Clause : Type :=
  | clause : set Atom -> Atom -> Clause.

Notation "ps ~> c" := (clause ps c) (at level 81).
\end{lstlisting}
\end{minipage}

Note also the \lstinline{Notation}-syntax, which allow us to define a custom notation,
making the code easier to read. The expression on the left-hand side of the \lstinline{:=} in quotation marks
is definitionally equal to the expression on the right-hand side in parentheses.
The level determines which notation should take precedence, with a higher level equaling a higher precedence.

We also want to model functions of the form $f : V \rightarrow N^{\infty}$.
We implement this in Coq using two types, \lstinline{Ninfty} and \lstinline{Frontier}.
\lstinline{Ninfty} is either a natural number or infinity.
\lstinline{Frontier} is a function from a string (variable) to \lstinline{Ninfty}.

\begin{minipage}{\linewidth}
\begin{lstlisting}[language=Coq, label={lst:ninfty_frontier_def}, caption={\lstinline{Ninfty} and \lstinline{Frontier} in Coq}]
Inductive Ninfty : Type :=
  | infty : Ninfty
  | fin   : nat -> Ninfty.

Definition Frontier := string -> Ninfty.
\end{lstlisting}
\end{minipage}

Using these definitions of \lstinline{Atom}, \lstinline{Clause} and \lstinline{Frontier},
we can define functions that check whether any given atom or clause is satisfied for any frontier.

\begin{minipage}{\linewidth}
\begin{lstlisting}[language=Coq, label={lst:atom_clause_true_def}, caption={\lstinline{atom_true} and \lstinline{clause_true} in Coq}]
Definition atom_true (a : Atom) (f : Frontier) : bool :=
  match a with
  | (x & k) =>
    match f x with
    | infty => true
    (* se explantation for <=? below *)
    | fin n => k <=? n
    end
  end.

Definition clause_true (c : Clause) (f : Frontier) : bool :=
  match c with
  | (conds ~> conc) =>
    if fold_right andb true (map (fun a => atom_true a f) conds)
    then (atom_true conc f)
    else true
  end.
\end{lstlisting}
\end{minipage}

The infix function \lstinline{<=?} is the boolean (and hence decidable) version of the
Coq function \lstinline{<=}, which uses \lstinline{Prop} and is not inherently decidable without
additional lemmas.

We can also define functions that "shift" the number value of atoms or whole clauses by some amount \lstinline{n : nat}.

\begin{minipage}{\linewidth}
\begin{lstlisting}[language=Coq, label={lst:shift_atom_clause_true}, caption={\lstinline{shift_atom} and \lstinline{shift_clause} in Coq}]
Definition shift_atom (n : nat) (a : Atom)  : Atom :=
  match a with
  | (x & k) => (x & (n + k))
end.

Definition shift_clause (n : nat) (c : Clause) : Clause :=
  match c with
  | conds ~> conc =>
    (map (shift_atom n) conds) ~> (shift_atom n conc)
  end.
\end{lstlisting}
\end{minipage}

Using these definitions, we can now define an important property that is
possible by Lemma 3.1 \cite[p. ~3]{mbezem}, since this lemma enables us to check whether or not a clause
is satisfied by a frontier for any shift of \lstinline{k : nat}.
We will use this property later to determine if a set of clauses is a valid model.

\begin{minipage}{\linewidth}
\begin{lstlisting}[language=Coq, label={lst:all_shifts_true}, caption={\lstinline{all_shifts_true} in Coq}]
Definition all_shifts_true (c : Clause) (f : Frontier) : bool :=
  match c with
  | (conds ~> conc) =>
      match conc with
      | (x & k) =>
          match f x with
          | infty => true
          | fin n => clause_true (shift_clause (n + 1 - k) c) f
          end
      end
  end.
\end{lstlisting}
\end{minipage}

\section{Semantic functions and predicates}

\subsection{The function \lstinline{sub_model}}

Given any set of clauses and a frontier (function assigning values to the variables),
we can determine if the frontier is a model of the set of clauses, i.e.
whether all shifts of all clauses are satisfied by the frontier.

We translate this property to Coq as the recursive function \lstinline{sub_model}.
We have two additional arguments \lstinline{V} and \lstinline{W}; these are
the set of variables (strings) from the set of clauses, and all , respectively.
The function \lstinline{vars_set_atom} simply returns all the variables used
in a set of atoms as a set of strings.

\begin{minipage}{\linewidth}
\begin{lstlisting}[language=Coq, label={lst:sub_model_def}, caption={The function \lstinline{sub_model} in Coq}]
Fixpoint sub_model (Cs : set Clause) (V W : set string) (f : Frontier) : bool :=
  match Cs with
  | []      => true
  | (l ~> (x & k)) :: t  =>
    (* conclusion not in W *)
    (negb (set_mem string_dec x W) ||
    (* some premise not in V *)
     negb (
       fold_right andb true
         (map (fun x => set_mem string_dec x V) (vars_set_atom l))
     ) ||
     all_shifts_true (l ~> (x & k)) f
    ) && sub_model t V W f
  end.
\end{lstlisting}
\end{minipage}

\subsection{The function \lstinline{geq}}

We want to determine whether all the values assigned to a set of variables
from one frontier are greater than or equal to all the values assigned to a set of variables
from another frontier. The values are of the type \lstinline{Ninfty}, and the function
only returns true if \textbf{all} the values from the first frontier are greater than
the values from the second frontier.

\begin{minipage}{\linewidth}
\begin{lstlisting}[language=Coq, label={lst:geq_def}, caption={Pointwise comparing frontiers with \lstinline{geq} in Coq}]
Fixpoint geq (V : set string) (g f : Frontier) : bool :=
  match V with
  | []      => true
  | h :: t  =>
    match g h with
    | infty => geq t g f
    | fin n =>
        match f h with
        | infty => false
        | fin k => (k <=? n) && geq t g f
        end
    end
  end.
\end{lstlisting}
\end{minipage}

\subsection{The predicate \lstinline{ex_lfp_geq}}
\label{ssec:ex_lfp_geq}

We can now combine \lstinline{sub_model} and \lstinline{geq} to construct a predicate
stating that there exists a frontier \lstinline{g} that is a model of the set of clauses \lstinline{Cs}
and is greater than or equal to another frontier \lstinline{f}.

\begin{minipage}{\linewidth}
\begin{lstlisting}[language=Coq, label={lst:ex_lfp_geq_def}, caption={\lstinline{ex_lfp_geq} in Coq, using both \lstinline{Prop} and \lstinline{Set}}]
Definition ex_lfp_geq_P (Cs : set Clause) (V W : set string) (f : Frontier) : Prop :=
  exists g : Frontier, geq V g f = true /\ sub_model Cs V W g = true.

Definition ex_lfp_geq_S (Cs : set Clause) (V W : set string) (f : Frontier) : Set :=
  sig (fun g : Frontier => prod (geq V g f = true) (sub_model Cs V W g = true)).
\end{lstlisting}
\end{minipage}

One thing to note here is that we can define this predicate either as having the type \lstinline{Prop} or as having the type \lstinline{Set}.
When defined with set \lstinline{Prop}, we define it as a logical predicate, using standard first-order logic syntax.
When defined with \lstinline{Set}, we define it as a type,
using the \lstinline{sig} type constructor in place of \lstinline{exists}.
We also use the \lstinline{prod} type constructor to represent the conjunction of two propositions.

The reason for defining \lstinline{ex_lfp_geq} with \lstinline{Set}, is that we can then use
Coq's extraction feature to generate Haskell code from the Coq definitions.
As briefly mentioned in \autoref{ssec:pat}, a proof can often be used to compute something.
In this case, we want to compute the actual frontier \lstinline{g} that satisfies the above predicate.
Coq distinguishes between logical objects (objects in \lstinline{Prop}) and infomative objects (objects in \lstinline{Set})
\cite[p. ~1-2]{coqextrnew}.
When extracting, Coq will remove as many logical objects as possible, meaning a predicate defined in \lstinline{Prop}
would simply be collapsed to \lstinline[language=Haskell]{()} (the unit type) in Haskell
\cite[p. ~8]{coqextroverview}.
Logical objects are only used to ensure correctness when constructing a proof in Coq,
and is not needed when actually computing something using the extracted code.

\section{The main proofs}

We have now laid the groundwork for the formalization of Theorem 3.2.
We precede the definition of Theorem 3.2 with two additional definitions, which helps us simplify its
definition and the proof of the theorem itself.

\subsection{The predicate \lstinline{pre_thm}}

Since (in our case) the formal definitions of lemma 3.3, which will be expanded on shortly,
and Theorem 3.2 share some structure, we define a proposition \lstinline{pre_thm}:

\begin{minipage}{\linewidth}
\begin{lstlisting}[language=Coq, label={lst:pre_thm_def}, caption={Def. of \lstinline{pre_thm}}]
Definition pre_thm (n m : nat) (Cs : set Clause) (V W : set string) (f : Frontier) :=
  incl W V ->
  Datatypes.length (nodup string_dec V) <= n ->
  Datatypes.length
    (set_diff string_dec
      (nodup string_dec V)
      (nodup string_dec W)
    ) <= m <= n ->
  ex_lfp_geq Cs (nodup string_dec W) (nodup string_dec W) f ->
  ex_lfp_geq Cs (nodup string_dec V) (nodup string_dec V) f.
\end{lstlisting}
\end{minipage}

\subsection{Lemma 3.3}

Lemma 3.3 from the paper \cite{mbezem} is used in the proof of Theorem 3.2 to solve \textcolor{red}{fill inn explanation here...}

We define it using \lstinline{pre_thm} as follows:

\begin{minipage}{\linewidth}
\begin{lstlisting}[language=Coq, label={lst:lem_33}, caption={Lemma 3.3 in Coq}]
Lemma lem_33 :
  forall Cs : set Clause,
  forall V W : set string,
  forall f : Frontier,
    (forall Cs' : set Clause,
     forall V' W' : set string,
     forall f' : Frontier,
     forall m : nat,
      pre_thm (Datatypes.length (nodup string_dec V) - 1) m Cs' V' W' f'
    ) ->
    incl W V ->
    ex_lfp_geq Cs (nodup string_dec W) (nodup string_dec W) f ->
    ex_lfp_geq Cs (nodup string_dec V) (nodup string_dec W) f.
Proof.
  (* ... *)
Qed.
\end{lstlisting}
\end{minipage}

\subsection{Theorem 3.2}

We can now formulate Theorem 3.2 using \lstinline{pre_thm}:

\begin{minipage}{\linewidth}
\begin{lstlisting}[language=Coq, label={lst:thm_32}, caption={Theorem 3.2 in Coq}]
Theorem thm_32 :
  forall n m : nat,
  forall Cs : set Clause,
  forall V W : set string,
  forall f : Frontier,
    pre_thm n m Cs V W f.
Proof.
  (* ... *)
Qed.
\end{lstlisting}
\end{minipage}

The proof of Theorem 3.2 is based on a primary induction on \lstinline{n} and a secondary induction on \lstinline{m}.

\subsubsection{Base case of \lstinline{n}}

The first base case is simple. We want to prove

(1) \tabto{2em}
\lstinline{ex_lfp_geq Cs (nodup string_dec V) (nodup string_dec V) f}.

Since \lstinline{n = 0}, we get that the length of \lstinline{V} is \lstinline{0},
and hence we get a new goal \lstinline{ex_lfp_geq Cs [] [] f}.
This is proven by the lemma \lstinline{ex_lfp_geq_empty},
which states that \lstinline{forall Cs f, ex_lfp_geq Cs [] [] f}.

\subsubsection{Inductive case of \lstinline{n}}

We start the inductive case of \lstinline{n} by doing a new induction on \lstinline{m}.

\subsubsection{Base case of \lstinline{m}}

The first base case is similar to the first base case of \lstinline{n}.
We again want to prove

\tabto{2em}
\lstinline{ex_lfp_geq Cs (nodup string_dec V) (nodup string_dec V) f}.

We now apply the lemma \lstinline{ex_lfp_geq_incl}, which states that

\tabto{2em}
\lstinline{forall Cs V W f, incl V W -> forall f, ex_lfp_geq Cs W W f -> ex_lfp_geq Cs V V f.}

We give this lemma the arguments of \lstinline{Cs}, \lstinline{nodup string_dec V} and \lstinline{nodup string_dec W}.
This generates to new goals,

$(1)$
\tabto{2em}
\lstinline{incl (nodup string_dec V) (nodup string_dec W)}

and

$(2)$
\tabto{2em}
\lstinline{ex_lfp_geq Cs (nodup string_dec W) (nodup string_dec W) f}.

The goal $(1)$ is proven by using a hypothesis that states that

\tabto{2em}
\lstinline{Datatypes.length (set_diff string_dec (nodup string_dec V) (nodup string_dec W)) <= m <= n}.

Since \lstinline{m = 0}, this means that
the set difference of \lstinline{V} and \lstinline{W} is empty.
We can now apply the lemma \lstinline{set_diff_nil_incl} on this hypothesis, which states that

\tabto{2em}
\lstinline{forall dec V W, set_diff dec V W = [] <-> incl V W.}

This gives us a hypothesis identical to our goal $(1)$, and therefore proves it.

The goal $(2)$ is proven by an existing hypothesis.

\subsubsection{Inductive case of \lstinline{m}}

\textcolor{red}{insert brief explanation here (from proof overview)...}

\section{Extraction to Haskell}

Using Coq's code extraction feature, we can extract Haskell code from our Coq definitions.

\begin{minipage}{\linewidth}
\begin{lstlisting}[language=Coq, label={lst:extraction}, caption={Extraction of Coq definitions to Haskell}]
Extraction Language Haskell.

Extract Constant map => "Prelude.map".
Extract Constant fold_right => "Prelude.foldr".

Extraction "/home/user/path/to/code/ex.hs"
  thm_32
  lem_33.
\end{lstlisting}
\end{minipage}

Coq will automatically determine definitions which depend on one another when doing extraction.
In the example above, we would not have needed to specify \lstinline{lem_33} to be extracted,
since \lstinline{thm_32} already depends on it.

By default, Coq will give its own implementation of any functions used, instead of using
Haskell's native implementations. If we want, we can specify what native Haskell functions
should be used when extracting a Coq function. In the example code above,
we specify that when extracting, \lstinline[language=Haskell]{Prelude.map} and \lstinline[language=Haskell]{Prelude.foldr}
should be used for the Coq functions \lstinline{map} and \lstinline{fold_right}.

In the next chapter we will go more into detail about the results of the extraction,
and the results of the Haskell code ran on some example input.
