\documentclass[a4paper,14pt]{article}

\usepackage{tabto}
\usepackage{bm}
\usepackage{amssymb}
\usepackage{amsmath}

\linespread{1.5}
\setlength{\parindent}{0em}

\title{Project Description}
\author{Andreas Salhus Bakseter}
\date{\today}

\begin{document}
\maketitle

\section{Formalization of mathematical problems}

When solving mathematical problems, one often uses proofs
to assert some claim. Proofs can be written either \textit{informally} or \textit{formally}.
\\

An informal proof is often written in a natural language, where the truth
of the proof is determined by if the reader is convinced by the proof or not.
ref (https://softwarefoundations.cis.upenn.edu/lf-current/Induction.html#lab58)

This can lead to mistakes in the proofs reasoning, as it is hard to be 100\%
certain that every step in the proof is solid (especially with larger ones).
ref (marc slides)


\section{Proof assistants}

A formal proof can be written like a computer program, where all the
arguments can be checked mechanically; often done using a proof assistant.

Coq is a proof assistant that enables us to write fomal proofs and verify them.
Coq uses type theory to verify proofs, but can also be used as a functional programming language.
ref (https://en.wikipedia.org/wiki/Coq)

Other examples of proof assistants include Agda, Isabelle, Lean and HOH.

\section{Type theory \& the Curry-Howard correspondance}



\section{Our case}

\section{Approach \& design choices}

\section{Implementation}

\section{Examples \& results}

\section{Evaluation}

\section{Conclusion}

\end{document}
