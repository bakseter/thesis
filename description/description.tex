\documentclass[a4paper,14pt]{article}

\usepackage{tabto}
\usepackage{bm}
\usepackage{amssymb}
\usepackage{amsmath}
\usepackage{biblatex}
\addbibresource{refs.bib}

\linespread{1.5}
\setlength{\parindent}{0em}

\title{\textbf{Project Description}}
\author{Andreas Salhus Bakseter}
\date{\today}

\begin{document}
\maketitle

\section{Formalization of mathematical problems}

When solving mathematical problems, one often uses proofs
to assert some claim. We can group proofs into two types; \textit{informal} and \textit{formal} proofs.
An informal proof is often written in a natural language, where the truth
of the proof is determined by if the reader is convinced by the proof or not.
\supercite{bpierce}
\\

As a proof grows larger and more complex, it becomes harder to follow, which can
ultimately lead to errors in the proofs reasoning. This might cause the whole proof
to be incorrect.
\supercite{rkhamsi}


\section{Proof assistants}

A formal proof can be written like a computer program, where all the
arguments can be checked mechanically; usually done with a \textit{proof assistant}.

Coq is a proof assistant that enables us to write fomal proofs and verify them.
Coq uses type theory to verify proofs, but can also be used as a functional programming language.
\supercite{cintro} Other examples of proof assistants include Agda, Isabelle, Lean and HOH.

\newpage

\section{Type theory \& propositions as types}

Type theory is used to create formal systems that group mathematical objects
with similar properties together by assigning them a "type".
% ref. (https://en.wikipedia.org/wiki/Type_theory)

Similarily to types in computer programming, we can use types to represent mathematical
objects. For example, we can use the type \texttt{nat} to represent natural nubers.
\\

The concept of propositions as types sees the proving of a mathematical proposition
as the same process as constructing a value of that type. For example, to prove a proposition $P$ whichs states "all integers are divisible by 2",
we must construct a value of the type $P$ that shows that this is true for all integers.
Since proofs are constructed using logical propositions, we can use this correspondance
to model a proof as a typed computer program.
The power of this concept comes from the fact that we can use a type checker to verify that
our program is typed correctly, and thus that the corresponding proof is valid.

\section{Our case}

We will use the Coq proof assistant to formalize parts of the proofs of the following paper, \textit{\citeauthor{mbezem}}.
This paper actually solves two problems that occur in dependant type systems where typings depend
on universe-level constraints.
\\

...more about the paper/theorems

% \section{Approach \& design choices}

% \section{Implementation}

% \section{Examples \& results}

% \section{Evaluation}

% \section{Conclusion}


\newpage
\printbibliography

\end{document}
