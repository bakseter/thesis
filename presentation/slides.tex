%\documentclass[handout]{beamer}
\documentclass[slides]{beamer}
% Vary the color applet  (try out your own if you like)
%\colorlet{structure}{red!20!black}
%\beamertemplateshadingbackground{yellow!20}{white}
%\usepackage{beamerthemeshadow}
%\usepackage[utf8x]{inputenc} CONFLICT!
%\usepackage[english,norsk,nynorsk]{babel}
\usepackage{tikz}
%\usetikzlibrary{trees}
\usepackage{bm}
\usepackage{caption}
\usepackage{multicol}
\usepackage{graphicx}
\usepackage{xcolor}
\usepackage{soul}
\makeatletter
\let\HL\hl
\renewcommand\hl{%
  \let\set@color\beamerorig@set@color
  \let\reset@color\beamerorig@reset@color
  \HL}
\makeatother
\usepackage{mathtools}
\graphicspath{ {./} }
\usepackage{listings}
\definecolor{dkgreen}{rgb}{0,0.6,0}
\definecolor{ltblue}{rgb}{0,0.4,0.4}
\definecolor{dkviolet}{rgb}{0.3,0,0.5}
\definecolor{dkblue}{rgb}{0,0,0.5}
\definecolor{dkred}{rgb}{0.5,0,0}
%\usepackage{enumitem}
\lstdefinelanguage{Coq}{
    tabsize=1,
    % Anything betweeen $ becomes LaTeX math mode
    mathescape=true,
    % Comments may or not include Latex commands
    texcl=false,
    % Vernacular commands
    morekeywords=[1]{Section, Module, End, Require, Import, Export,
        Variable, Variables, Parameter, Parameters, Axiom, Hypothesis,
        Hypotheses, Notation, Local, Tactic, Reserved, Scope, Open, Close,
        Bind, Delimit, Definition, Let, Ltac, Fixpoint, CoFixpoint, Add,
        Morphism, Relation, Implicit, Arguments, Unset, Contextual,
        Strict, Prenex, Implicits, Inductive, CoInductive, Record, Fail,
        Structure, Canonical, Coercion, Context, Class, Global, Instance,
        Program, Infix, Theorem, Lemma, Corollary, Proposition, Fact,
        Remark, Example, Proof, Goal, Save, Qed, Defined, Hint, Resolve,
        Rewrite, View, Search, Show, Print, Printing, All, Eval, Check,
        Projections, inside, outside, Def, Extract, Extraction, Constant},
    % Gallina
    morekeywords=[2]{forall, exists, exists2, fun, fix, cofix, struct,
        match, with, end, as, in, return, let, if, is, then, else, for, of,
        nosimpl, when},
    % Sorts
    morekeywords=[3]{Type, Prop, Set, true, false, option},
    % Various tactics, some are std Coq subsumed by ssr, for the manual purpose
    morekeywords=[4]{pose, set, move, case, elim, apply, eapply, clear, hnf,
        intro, intros, generalize, rename, pattern, after, destruct,
        induction, using, refine, inversion, injection, rewrite, congr,
        unlock, compute, ring, field, fourier, replace, fold, unfold,
        change, cutrewrite, simpl, have, suff, wlog, suffices, without,
        loss, nat_norm, assert, cut, trivial, revert, bool_congr, nat_congr,
        symmetry, transitivity, auto, split, left, right, autorewrite},
    % Terminators
    morekeywords=[5]{by, done, exact, reflexivity, tauto, romega, omega,
        assumption, solve, contradiction, discriminate},
    % Control
    morekeywords=[6]{do, last, first, try, idtac, repeat},
    % Comments delimiters, we do turn this off for the manual
    morecomment=[s]{(*}{*)},
    % Spaces are not displayed as a special character
    showstringspaces=false,
    % String delimiters
    morestring=[b]",
    morestring=[d],
    % Size of tabulations
    tabsize=2,
    % Enables ASCII chars 128 to 255
    extendedchars=false,
    % Case sensitivity
    sensitive=true,
    % Automatic breaking of long lines
    breaklines=false,
    % Default style fors listings
    basicstyle=\small,
    % Position of captions is bottom
    captionpos=b,
    % flexible columns
    columns=[l]flexible,
    % Style for (listings') identifiers
    identifierstyle={\ttfamily\color{black}},
    % Style for declaration keywords
    keywordstyle=[1]{\ttfamily\color{dkviolet}},
    % Style for gallina keywords
    keywordstyle=[2]{\ttfamily\color{dkgreen}},
    % Style for sorts keywords
    keywordstyle=[3]{\ttfamily\color{ltblue}},
    % Style for tactics keywords
    keywordstyle=[4]{\ttfamily\color{dkblue}},
    % Style for terminators keywords
    keywordstyle=[5]{\ttfamily\color{dkred}},
    %Style for iterators
    %keywordstyle=[6]{\ttfamily\color{dkpink}},
    % Style for strings
    stringstyle=\ttfamily,
    % Style for comments
    commentstyle={\ttfamily\color{dkgreen}},
    %moredelim=**[is][\ttfamily\color{red}]{/&}{&/},
    literate=
    {\\forall}{{\color{dkgreen}{$\forall\;$}}}1
    {<-}{{$\leftarrow\;$}}1
    {=>}{{$\Rightarrow\;$}}1
    {==}{{\code{==}\;}}1
    {==>}{{\code{==>}\;}}1
    %    {:>}{{\code{:>}\;}}1
    {->}{{$\rightarrow\;$}}1
    {–>}{{$\rightarrow\;$}}1
    {<->}{{$\leftrightarrow\;$}}1
    {<==}{{$\leq\;$}}1
    {\#}{{$^\star$}}1
    {\\o}{{$\circ\;$}}1
    {\@}{{$\cdot$}}1
    {\/\\}{{$\wedge\;$}}1
    {\\\/}{{$\vee\;$}}1
    {++}{{$++\;$}}1
    {+++}{{$+++\;$}}1
    {--}{{- - }}1
    {---}{{- - - }}1
    {\@\@}{{$@$}}1
    {\\mapsto}{{$\mapsto\;$}}1
    {\\hline}{{\rule{\linewidth}{0.5pt}}}1
    {~>}{{$\leadsto\;$}}1
    {<=}{{$\leq\;$}}1
    {~}{{$\sim$}}1
    %
}[keywords,comments,strings]

\lstset{
    basicstyle=\linespread{0.8}\footnotesize\ttfamily,
    keywordstyle=\color{red},
    showstringspaces=false,
    stringstyle=\color{blue},
    commentstyle=\color{gray},
    breakatwhitespace=true,
    breaklines=true,
    postbreak=\mbox{\textcolor{violet}{$\hookrightarrow$}\space},
    tabsize=4,
    language=Coq,
    xleftmargin=0.5cm,
    abovecaptionskip=0.5cm
}

%\usepackage[all]{xy}
%\usepackage{multicol}

%\setbeamertemplate{navigation symbols}{}++++++
%\setbeamertemplate{footline}[frame number]
\usetheme{Montpellier}

\title{A Case Study in Dependent Type Theory:\\ Extracting a Certified Program from the Formal Proof of its Specification}
\author{Andreas Salhus Bakseter}
\institute{Department of Informatics\\University of Bergen}

\begin{document}

\frame{\titlepage}

\section{Overview}

\frame {
    \frametitle{Overview}

    \begin{enumerate}
        \item Background
        \item The Case
        \item Approach \& Design Choices
        \item Implementation
        \item Examples \& Results
        \item Evaluation
        \item Related \& Future Work
        \item Conclusion
    \end{enumerate}
}

\section{Background}
\subsection{Proofs}

\begin{frame}[fragile]
    \frametitle{Proofs}

    \begin{itemize}
        \item Important part of mathematics
        \item
            \textbf{Informal proof}: written in a natural language, for humans.

            \begin{quote}
              "Any natural number $n \in \mathbb{N}$ is either even or odd."
            \end{quote}

        \item \textbf{Formal proof}: written in a formal language, for computers.
            \begin{lstlisting}
Theorem even_or_odd :
    forall n : nat,
    exists k : nat, n = 2 * k \\/ n = 2 * k + 1.
            \end{lstlisting}
    \end{itemize}
\end{frame}

\subsection{Type Theory}

\frame{
    \frametitle{Type Theory}

    \begin{itemize}
        \item A foundation of mathematics
        \item Every mathematical object has a type, $a : A$
        \item We must know the construction of every elemtent of a type
        \item Provides us with rules of inference for manipulating types \& objects
    \end{itemize}
}

\subsection{Propositions as types}

\frame{
    \frametitle{Propositions as types}

    \begin{itemize}
        \item Propositions are also types, and their elements are proofs
        \item Since we know the construction of any element of a type,
            we know the construction of any proof of a proposition
        \item Also means that proofs are programs, i.e. they can be executed
    \end{itemize}
}

\subsection{Dependent types}

\frame{
    \frametitle{Dependent types}

    \begin{itemize}
        \item Types can depend on objects
        \item Gives us more expressive power, especially when defining propositions using types
        \item Using $\Sigma$-types we can model existential quantification
        \item Using $\Pi$-types we can model universal quantification
    \end{itemize}

}

\subsection{Proof Assistants}

\frame{
    \frametitle{Proof Assistants}

    \begin{itemize}
        \item Software tools for constructing \& verifying formal proofs
        \item Based on concepts from type theory, propositions as types
        \item \textbf{Coq}:
            \begin{itemize}
                \item based on the type theory \textit{Calculus of Inductive Constructions}
                \item uses \textit{Gallina} as its specification language
                \item uses \textit{Ltac} as its tactic language
                \item supports extraction of programs
            \end{itemize}
    \end{itemize}
}

\section{The Case in Question}
\subsection{Overview of case}

\frame{
    \frametitle{Overview of case}

    Two problems in lattice theory soled by Bezem \& Coquand:

    \begin{enumerate}
        \item the uniform word problem for an equational theory of semilattices with
            one inflationary endomorphism can be solved in polynomial time
        \item loop-checking in a type system with universe-level constratins can be done
            in polynomial time
    \end{enumerate}
}

\subsection{Research questions}

\frame{
    \frametitle{Research questions}

    We want to answer the following questions:

    \begin{enumerate}
        \item Are the results from Bezem \& Coquand correct?
        \item Is it feasible to formalize such complex proofs as in Bezem \& Coquand?
        \item Is the formalization process worth the effort, i.e. what do we gain from it?
    \end{enumerate}
}

\subsection{Relevant parts of the paper}
\frame{
    \frametitle{Relevant parts of the paper}

    \begin{itemize}

        \item \textbf{join-semilattices}:
            partially ordered set where any two elements have a least upper bound,
            called their join, denoted $\lor$
        \item \textbf{inflationary endomorphism}:
            function that maps an element to itself or to a greater element in the ordered set,
            denoted $\_^+$
        \item \textbf{semilattice presentation p1}:
            Set $V$ of variables and set $C$ of constraints.
            For a semilattice term $t \in V$ and $k \in \mathbb{N}$, $t + 0 = t$ and $t + 1 = t^+$.
            A term over $V$ has the form $x_1 + k_1 \lor ... \lor x_m + k_m$, where $x_i \in V$ and $k_i \in \mathbb{N}$.
    \end{itemize}
}

\frame{
    \frametitle{Relevant parts of the paper}

    \begin{itemize}
        \item \textbf{Horn clauses}:
            Propositional clauses $A \rightarrow b$ with a non-empty body
            of atoms $A$ and a conclusion $b$. Atoms are of the form $x + k$,
            where $x \in V$ and $k \in \mathbb{N}$.
        \item \textbf{semilattice presentation p2}:
            A constraint over $V$ has the form $s = t$, where $s$ and $t$ are terms over $V$.
            For a constraint s = t, we can generate Horn clauses by replacing join by conjuction,
            denoted by "," and $\leq$ by implication.
            The set of all generated clauses from a constraint is denoted by $S_{s=t}$.
            For a set of constraints $C$, we define $S_C = \bigcup_{s=t \in C} S_{s=t}$.
    \end{itemize}
}

\frame{
    \frametitle{Relevant parts of the paper}

    \textbf{Example of generating clauses from constraints}:
        \begin{enumerate}
            \item
                $a \lor b = c^+ \xRightarrow{\text{axiom}}$
                $a \lor b \leq c^+$ and $c^+ \leq a \lor b$.
            \begin{enumerate}
                \item
                    $a \lor b \leq c^+ \xRightarrow{\text{generate}}$ \hl{$a, b \rightarrow c^+$}
                \item
                    $c^+ \leq a \lor b \xRightarrow{\text{axiom}} c^+ \leq a$ and $c^+ \leq b$.

                    \begin{enumerate}
                        \item
                            $c^+ \leq a \xRightarrow{\text{generate}}$ \hl{$c^+ \rightarrow a$}

                        \item
                            $c^+ \leq b \xRightarrow{\text{generate}}$ \hl{$c^+ \rightarrow b$}
                    \end{enumerate}
            \end{enumerate}
        \end{enumerate}

    We are then left with $S_{a \lor b = c^+} = \{a, b \rightarrow c^+, c^+ \rightarrow a, c^+ \rightarrow b\}$.
}

\frame{
    \frametitle{Relevant parts of the paper}

    \begin{itemize}
        \item
            Horn clause provability $\Longleftrightarrow$ semilattice provaibility

        \item
            \textbf{closure under shifting upwards}:
            if $A \rightarrow b$ is in a set of clauses, then so is
            $A + 1 \rightarrow b + 1$, where $A + 1$ is every atom in $A$ shifted upwards by $1$.

        \item
            $\bm{\overline{S_C}}$: the smallest subset of $S_C$ that is closed under shifting upwards.
        \item
            $\bm{\overline{S_C}\mid W}$: the set of clauses in $\overline{S_C}$ with only variables from $W$.
        \item
            $\bm{\overline{S_C}\downarrow W}$: the set of clauses in $\overline{S_C}$ with conclusion over $W$.

        \item
            A model of a set of clauses $\overline{S_C}$ is defined by a function $f : V \rightarrow \mathbb{N}^\infty$,
            where $\mathbb{N}^\infty$ is the set of natural numbers extended with $\infty$.
    \end{itemize}
}

\frame{
    \frametitle{Relevant parts of the paper}

    \begin{itemize}
        \item
            \textbf{Lemma 3.1}: provability of Horn clauses is decidable
        \item
            \textbf{Theorem 3.2}: we can compute the least model of a set of clauses $\overline{S_C}$
        \item
            \textbf{Lemma 3.3}: given some assumptions, we can compute the least models of $\overline{S_C}\mid W$ and $\overline{S_C}\downarrow W$

    \end{itemize}
}

\section{Approach \& Design Choices}
\subsection{Simplifications}

\frame{
    \frametitle{Simplifications}

    \begin{enumerate}
        \item
            incomplete formal proofs
        \item
            leaving out formal proof for Lemma 3.3
        \item
            proof of minimality of least model
    \end{enumerate}
}


\subsection{Modeling sets in Coq}

\frame{
    \frametitle{Modeling sets in Coq}

    \begin{itemize}
        \item
            \lstinline{List} \& \lstinline{ListSet}
            \begin{itemize}
                \item
                    uses lists, familiar and easy, inductive
                \item
                    has order and duplicates, can be combatted
                \item
                    type polymorphic, requires decidable equality
            \end{itemize}
        \item
            \lstinline{MSetWeakList}
            \begin{itemize}
                \item
                    a lot more complicated, module functor
                \item
                    requires boilerplate code for each type
            \end{itemize}
        \item
            \lstinline{Ensembles}
            \begin{itemize}
                \item
                    uses inductive propositions, i.e.: $x \in A \Rightarrow x \in A \cup B$
                \item
                    cannot reason about set size, ergo useless for us
            \end{itemize}
    \end{itemize}

    We went with \lstinline{List} \& \lstinline{ListSet}.
}

\section{Implementation}
\subsection{Data types}

\begin{frame}[fragile]
    \frametitle{Data types}

    \begin{itemize}
        \item
            \textbf{Atom}: \lstinline[language=Coq]{string -> nat -> Atom}, e.g. \lstinline[language=Coq]{"x" & 0}
        \item
            \textbf{Clause}: \lstinline[language=Coq]{set Atom -> Atom -> Clause}, e.g. \lstinline[language=Coq]{["x" & 0; "y" & 1] ~> "z" & 2]}
        \item
            \textbf{Ninfty}: either a natural number \lstinline{fin n} or infinity \lstinline{infty}
        \item
            \textbf{Frontier}: \lstinline{string -> Ninfty}
    \end{itemize}
\end{frame}

\begin{frame}[fragile]
    \frametitle{Data types}

    \begin{itemize}
        \item
            \textbf{check if satisfied by frontier}: \lstinline{atom_true}, \lstinline{clause_true}
        \item
            \textbf{shift by n upwards}: \lstinline{shift_atom}, \lstinline{shift_clause}
        \item
            \textbf{satisfied for all shifts}: \lstinline{all_shfits_true}
            \begin{itemize}
                \item
                    possible by Lemma 3.1
                \item
                    will be used to determine model
            \end{itemize}
    \end{itemize}
\end{frame}

\subsection{Functions \& predicates}

\begin{frame}[fragile]
    \frametitle{Functions \& predicates}

    \begin{itemize}
        \item \begin{minipage}{\linewidth}\lstinline{sub_model Cs V W f : bool}\end{minipage}
            For each clause in \lstinline{Cs}, checks three conditions:
            \begin{enumerate}
                \item
                    variable in conclusion of clause is not in \lstinline{W}
                \item
                    some variable in the premise of the clause is not in \lstinline{V}
                \item
                    the clause is true for all shifts
            \end{enumerate}
            If any of these are true for all clauses, \lstinline{f} is a model of \lstinline{Cs}.

            Can represent either $\overline{S_C}$, $\overline{S_C}\mid W$ or $\overline{S_C}\downarrow W$.
        \item \begin{minipage}{\linewidth}\lstinline{sub_forward Cs V W f : set string * Frontier}\end{minipage}
            For each clause in \lstinline{Cs}, checks the same as \lstinline{sub_model},
            and adds variable in conclusion to set of "improvable" variables if all are false.
            Returns tuple where first value is the set of improvable variables,
            and the second value is a new frontier that is incremented by one for each variable in the set.
    \end{itemize}
\end{frame}

\begin{frame}[fragile]
    \frametitle{Functions \& predicates}

    \begin{itemize}
        \item \begin{minipage}{\linewidth}\lstinline{geq V g f : bool}\end{minipage}
            Checks if frontier \lstinline{g} is greater than or equal to frontier \lstinline{f} for all variables in \lstinline{V}.
        \item \begin{minipage}{\linewidth}\lstinline{ex_lfp_geq Cs V W f : Set :=}\end{minipage}
              \begin{minipage}{\linewidth}\lstinline{    exists g : Frontier, geq V g f /\ sub_model Cs V W g.}\end{minipage}
    \end{itemize}
\end{frame}

\subsection{Theorem 3.2}

\begin{frame}[fragile]
    \frametitle{Theorem 3.2}

    \begin{minipage}{\linewidth}\lstinline{Theorem thm_32 :}\end{minipage}
    \begin{minipage}{\linewidth}\lstinline{  forall n m Cs V W f,}\end{minipage}
    \begin{minipage}{\linewidth}\lstinline{    incl W V ->}\end{minipage}
    \begin{minipage}{\linewidth}\lstinline{    Datatypes.length (nodup string_dec V) <= n ->}\end{minipage}
    \begin{minipage}{\linewidth}\lstinline{    Datatypes.length}\end{minipage}
    \begin{minipage}{\linewidth}\lstinline{      (set_diff string_dec}\end{minipage}
    \begin{minipage}{\linewidth}\lstinline{        (nodup string_dec V)}\end{minipage}
    \begin{minipage}{\linewidth}\lstinline{        (nodup string_dec W)}\end{minipage}
    \begin{minipage}{\linewidth}\lstinline{    ) <= m <= n ->}\end{minipage}
    \begin{minipage}{\linewidth}\lstinline{    ex_lfp_geq Cs (nodup string_dec W) (nodup string_dec W) f ->}\end{minipage}
    \begin{minipage}{\linewidth}\lstinline{    ex_lfp_geq Cs (nodup string_dec V) (nodup string_dec V).}\end{minipage}
\end{frame}

\subsection{Theorem 3.2 proof overview}
\frame{
    \frametitle{Theorem 3.2 proof overview}

    \begin{itemize}
        \item
            proof by primary induction on \lstinline{n} (length of \lstinline{V}),
            and secondary induction on \lstinline{m} (length of \lstinline{W}).
            Both base cases and induction step of \lstinline{n} relatively easy,
            induction step of \lstinline{m} by far most complex.

        \item
            In induction step of \lstinline{m}, we distinguish between three cases:
            \begin{enumerate}
                \item
                    $W = \emptyset$
                \item
                    $W = V$
                \item
                    $\emptyset \subset W \subset V$,
                    apply primary and secondary induction hypotheses,
                    and apply Lemma 3.3
            \end{enumerate}
    \end{itemize}
}

\section{Examples \& Results}
\subsection{Extraction and output}

\frame{
    \frametitle{Extraction and output}

    \begin{itemize}
        \item
            We can extract formal proofs to Haskell code,
            and run Theorem 3.2 on a set of clauses.
        \item
            \begin{minipage}{\linewidth}\lstinline{Example Cs := [ ["a" & 0] ~> "b" & 1; ["b" & 1] ~> "c" & 2 ]}\end{minipage}
            \begin{minipage}{\linewidth}\lstinline{thm_32 Cs "a" => Fin 0}\end{minipage}
            \begin{minipage}{\linewidth}\lstinline{thm_32 Cs "b" => Fin 1}\end{minipage}
            \begin{minipage}{\linewidth}\lstinline{thm_32 Cs "c" => Fin 2}\end{minipage}
        \item
            \begin{minipage}{\linewidth}\lstinline{Example Cs_loop := [ ["a" & 0] ~> "a" & 1 ]}\end{minipage}
            \begin{minipage}{\linewidth}\lstinline{thm_32 Cs_loop "a" => Infty}\end{minipage}
    \end{itemize}
}

\subsection{Real-world example \& limitations}

\frame{
    \frametitle{Real-world example \& limitations}

    \begin{itemize}
        \item
            We can use Theorem 3.2 algorithm to check type universe level consistency in Coq,
            which Coq does when interpreting/compiling.
        \item
            The Coq Command \lstinline{Print Universes} gives us a list of universe constraints,
            can be translated to our syntax. In testing this produced correct
            type universe levels for over 5000 constraints.
        \item
            Algorithm does not always work due to Lemma 3.3 not formally proven,
            i.e. fails when extending previous example with
            \begin{minipage}{\linewidth}\lstinline{Example Cs := [ (*...*) ["c" & 2] ~> "d" & 3 ].}\end{minipage}
    \end{itemize}
}

\section{Evalution}

\frame{
    \frametitle{Evalution}

    \begin{enumerate}
        \item Are the results correct?
            \begin{itemize}
                \item we have not formalized every part of results,
                    and we have simplified some proofs
                \item but, with a full formal proof of Theorem 3.2, we can be very confident that the results are correct
            \end{itemize}
        \item Is it feasible to formalize such complex proofs?
            \begin{itemize}
                \item by our effort, yes
                \item could be done faster and/or better with more experience
            \end{itemize}
        \item Is the formalization process worth the effort?
            \begin{itemize}
                \item
                    we gain a useable protoype
                \item
                    can be used as a starting point for more efficient implementations
            \end{itemize}
    \end{enumerate}
}

\section{Related \& Future Work}
\subsection{Related work}

\frame{
    \frametitle{Related work}

    Ongoing effort by Matthieu Sozeau, Coq Team to implement version of algorithm for use
    in universe consitencty checking in Coq.
    Main focus on speed, and only supports clauses of the form
    $x + k \geq y$, where $k \in \{0, 1\}$.

}

\subsection{Future work}

\frame{
    \frametitle{Future work}

    \begin{enumerate}
        \item
            Complete proofs of remaining logical lemmas
        \item
            Formal proof of Lemma 3.3
        \item
            Prove minimality of model in Theorem 3.2
    \end{enumerate}
}

\section{Conclusion}

\frame{
    \frametitle{Conclusion}


}

\end{document}
