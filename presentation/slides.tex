%\documentclass[handout]{beamer}
\documentclass[slides]{beamer}
% Vary the color applet  (try out your own if you like)
%\colorlet{structure}{red!20!black}
%\beamertemplateshadingbackground{yellow!20}{white}
%\usepackage{beamerthemeshadow}
%\usepackage[utf8x]{inputenc} CONFLICT!
%\usepackage[english,norsk,nynorsk]{babel}
\usepackage{tikz}
%\usetikzlibrary{trees}
\usepackage{bm}
\usepackage{caption}
\usepackage{multicol}
\usepackage{graphicx}
\graphicspath{ {./} }
\usepackage{listings}
%\usepackage{enumitem}

%\usepackage[all]{xy}
%\usepackage{multicol}

%\setbeamertemplate{navigation symbols}{}++++++
%\setbeamertemplate{footline}[frame number]
\usetheme{Montpellier}

\title{A Case Study in Dependent Type Theory:\\ Extracting a Certified Program from the Formal Proof of its Specification}
\author{Andreas Salhus Bakseter}
\institute{Department of Informatics\\University of Bergen}

\begin{document}

\frame{\titlepage}

\section{Overview}

\frame {
    \frametitle{Overview}

    \begin{enumerate}
        \item Background
        \item The Case
        \item Approach \& Design Choices
        \item Implementation
        \item Examples \& Results
        \item Evaluation
        \item Related \& Future Work
        \item Conclusion
    \end{enumerate}
}

\section{Background}
\subsection{Proofs}

\frame {
    \frametitle{Proofs}

    \begin{itemize}
        \item<1-> proofs are an important part of mathematics
        \item<2-> two kinds of proofs:
            \begin{itemize}
                \item<3-> \textbf{informal proofs}, natural language, by humans for humans
                \item<4-> \textbf{formal proofs}, formal language, hard for humans easy for computers
            \end{itemize}
        \item<5-> we can tranfsorm informal proofs into formal proofs (formalization)
        \item<6-> using computers \& proof assistants we can check formal proofs (verification)
        \item<7-> we can also extract programs from formal proofs
    \end{itemize}
}

\subsection{(Dependent) Type Theory \& Propositions as Types}

\frame{
    \frametitle{(Dependent) Type Theory \& Propositions as Types}

    \begin{itemize}
        \item<1-> foundation of mathematics
        \item<2-> every object has a type, $a : A$, must know its construction
        \item<4-> provides us with rules of inference for manipulating types \& objects
        \item<5->\textbf{propositions as types}:
            \begin{itemize}
                \item<6-> propositions are types
                \item<7-> proofs are objects
                \item<8-> proof of a proposition is an object of that type
            \end{itemize}

        \item<9-> \textbf{dependent types}:
            \begin{itemize}
                \item<10-> gives us more expressive types
                \item<11-> $\Sigma$-types to model existential quantification
                \item<12-> $\Pi$-types to model universal quantification
            \end{itemize}
    \end{itemize}
}

\subsection{Proof Assistants}

\frame{
    \frametitle{Proof Assistants}

    \begin{itemize}
        \item<1-> software tools for constructing \& verifying formal proofs
        \item<2-> based on type theory, propositions as types
        \item<3-> \textbf{Coq}:
            \begin{itemize}
                \item<4-> based on the type theory \textit{Calculus of Inductive Constructions}
                \item<5-> uses \textit{Gallina} as its specification language
                \item<6-> uses \textit{Ltac} as its tactic language, for ease of use
                \item<6-> supports extraction of programs
            \end{itemize}
    \end{itemize}
}

\section{The Case in Question}
\subsection{Overview of case}

\frame{
    \frametitle{Overview of case}

    \begin{itemize}
        \item<1->two problems in lattice theory solved by Bezem \& Coquand
        \item<2->we want to answer these questions:
            \begin{itemize}
                \item<3->are the results from Bezem \& Coquand correct?
                \item<4->is it feasible to formalize complex proofs, such as these?
                \item<5->is the formalization process worth the effort?
            \end{itemize}
    \end{itemize}
}

\frame{
    \frametitle{Relevant parts of the paper}

    \begin{itemize}
        \item<1->join-semilattices: partially ordered set where any two elements have a least upper bound,
            called their join, denoted $\lor$
        \item<2->
    \end{itemize}
}

\end{document}
