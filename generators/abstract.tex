\pagenumbering{roman}

\begin{abstract}

    Proofs are an important part of mathematics, but they are not without their flaws.
    Most proofs are written by humans, and humans make mistakes.
    In this thesis, we explore the use of proof assistants to construct formal versions of informal proofs,
    and to extract certified programs from these formal proofs.
    We study a specific case of two problems from dependent type theory,
    solved by \citeauthor{mbezem} in \cite{mbezem}.
    Firstly, we ask ourselves: are the results from \cite{mbezem} correct?
    The informal proofs posed in \cite{mbezem}, used to justify the correctness of the results,
    are complex enough that mistakes are possible.
    Using the Coq proof assistant, we formalize parts of the proofs from \cite{mbezem},
    to gain more confidence in the correctness these informal proofs.
    Secondly, is the process of formalization a feasible one?
    Transforming an informal proof into a formal proof is not necessarily an easy or straightforward process,
    and there are several approaches to formalization of a proof.
    Lastly, is the process of formalization worth the effort?
    A fully formalized proof gives us almost complete confidence in the correctness of the proof,
    and also has the added bonus of extracting a certified program from the proof.
    In our case, this algorithm has some practical use cases in a real-world setting.

\end{abstract}

\renewcommand{\abstractname}{Acknowledgments}
\begin{abstract}
    \lipsum[1]

	\vspace{1cm}
	\hspace*{\fill}\texttt{Andreas Salhus Bakseter}\\
	\hspace*{\fill}\today
\end{abstract}
\setcounter{page}{1}
\newpage
