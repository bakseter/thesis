\pagenumbering{roman}

\begin{abstract}

    Proofs are an important part of mathematics, but they are not without their flaws.
    Proofs are written by humans, and humans make mistakes.
    In this thesis, we explore the use of proof assistants to construct formal versions of informal proofs,
    and to extract certified programs from these formal proofs.
    We study a specific case of two problems from dependent type theory,
    solved by \citeauthor{mbezem} in \cite{mbezem}.
    Firstly, we ask ourselves: is the informal proof of \Cref{thm:3.2} from \cite{mbezem} correct?
    \Cref{thm:3.2} is a complex enough proof such that mistakes are possible.
    Using the Coq proof assistant, we formalize the proof of \Cref{thm:3.2} from \cite{mbezem},
    to gain more confidence in the correctness of the informal proof.
    Secondly, is the process of formalization worth the effort?
    Transforming an informal proof into a formal proof is not necessarily an easy or straightforward process,
    and there are several approaches to formalization of a proof.
    Lastly, is the the certified program extracted from the formal proof of \Cref{thm:3.2} useful?
    The algorithm extracted from the proof has practical applications in some fields of computer science,
    and we explore the usefulness of the algorithm in a real-world setting.


\end{abstract}

\renewcommand{\abstractname}{Acknowledgements}
\begin{abstract}
    \lipsum[1]

	\vspace{1cm}
	\hspace*{\fill}\texttt{Andreas Salhus Bakseter}\\
	\hspace*{\fill}\today
\end{abstract}
\setcounter{page}{1}
\newpage
